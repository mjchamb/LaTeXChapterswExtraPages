%Acknowledgments

\setlength\epigraphwidth{8cm}
\setlength\epigraphrule{0pt}

\chapter{Acknowledgments}\label{Acknowledgments}

I derived a great deal of pleasure writing this text, and would be remiss to imagine myself an island. The following people (those I know and those I don't) deserve my thanks.\\

First and foremost, thanks my editor, Bruce Smith.\footnote{Available for contract editing at \url{https://www.linkedin.com/in/brucesmith9/}} I'm not an innately skilled writer, nor have I spent much time honing the skill outside this book. Knowing what the first draft was like, he's a miracle worker.\\ 

To my family, who have always been there for me, thank you.\\

To my chosen family (friends), I realize the impact and impression you've had on me.\\

\epigraph{\emph{"If I'm the sum of all my friends, then all my friends are some of me”}}{---The Matches, \emph{You (Don't) Know Me},\\ from the album \emph{Decomposer}}

To anyone I have been lucky enough to count among my friends throughout life, I give my perpetual gratitude. Special thanks to the longtime emotional support pillars and life decision consultants: August Brunsman, Joe Camerlengo, Michael Donohoe, Katie Gritti, Sarah Hammond, Adam Kucharski, Amanda Kay Metskas, and Brian Rutledge.\\    

Special thanks to Trisha Kennedy, Shelley Lang, and Heather McConnell for getting me through pharmacy school. Class would have been unbearably boring without you guys. Back Row Ki was wildly successful in retrospect.\\

And in the spirit of naive optimism and relentless positivity, thanks to all the people I'm not huge fans of for providing me valuable life lessons and examples of what not to be, \emph{via negativa}.\\

To all my intellectual idols and inspirations that follow, thank you for making me smarter by allowing me to see through the lens of someone thinking on a different wavelength. Life is complicated, and it helps to have more blind men grasping at the elephant.\\

First, many thanks to Pete Adeney (aka Mr. Money Mustache). My only wish was that I had found you sooner. I hope to pay a visit to the co-working space in Longmont as an early retiree someday.\\

I owe a debt of gratitude to Rudy from Alpha Investments. He runs a Youtube channel detailing mainly his long term investments in Magic the Gathering cards and other collectibles. But it’s secretly a channel about financial education, investing, and how to truly grow wealth. I personally have never owned MTG cards, though I plan on getting some sealed product once I have more disposable passive income. I haven't played an in-person MTG game either,\footnote{I say this so that anyone reading who is is not interested in MTG will hopefully still be willing to soak up as much knowledge as they can from his channel.} but always love and recommend his videos to anyone interested in investing.

He may look like a hairy guy that lives in a basement and can’t afford a haircut, but he’s got professional finance experience and an accurate view of the way the world works. More importantly, he’s a genuine guy who is willing to share what he knows in an attempt to educate and help people.

My personal admiration for Rudy stems from one specific video, “Timmy’s are Losers.” One of the first Alpha Investments videos I ever watched, it detailed the difficulty of going long options. I knew everything he covered at an intellectual level, but I saw it at precisely the right time in my life, namely when the tide had turned on my overall P\&L for all my active investing and options trading adventures. It really made me realize how hard the game was, and the mindset I was functioning in. Immediately after that, I watched “StoryTime: I almost BANKRUPTED Alpha Investments,” and it was just what I needed to hear.

\begin{center}
\fbox
{
\begin{minipage}{0.90\textwidth}
    \textbf{Admitting you were wrong and changing is\\ simultaneously the hardest thing and the most\\ important thing to do in life.} 
\end{minipage}
} 
\end{center}


To be clear, I took a very measured bet after explicit, extended deliberation in the hopes I could speed up the clock on FIRE, since I had come to the idea later in life. My decision was influenced by my acute awareness of just how short life is (thanks to my day job at a cancer hospital), and an admittedly unhealthy amount of hubris. But realizing I was in the Timmy mindset (making bets, thinking only of upside, wanting a big win to happen) and that everyone makes mistakes that they can (hopefully) move on from really helped shake me out of the delusion that I was going to be clever enough to beat the market and make endless long-term profits buying options and trading spreads. 

Selling covered calls is easy, buying long calls is pure risk. Alpha Investments gave me the nudge I needed to get me serious and committed to wealth accumulation and FIRE. For that, I owe The Rudy a taco or three.

Other amazing videos of his include “CEOs of Brokerage firms do not own mutual funds,” “Timmy Learns about Credit Card Manipulation,” and “Royal Caribbean Cruise Lines: Let’s Talk about Bonds.” Basically any time you see a whiteboard or Rudy wearing a pink 80’s visor you’re about to get a free education worth more than most MBA programs. 

At the time I’m writing this, over to 2 years into the writing process and on the final edits of this text, Rudy just released a video “Wealth and Poor People”, which describes, among other things like margin equity loans,\footnote{Mr. Money Mustache recently discovered these and wrote about using one to buy a home for cash in the January 29th, 2021 post titled “The Margin Loan: How to Make a \$400,00 Impulse Purchase”} how very wealthy people will sometimes sell covered calls against their holdings to collect roughly a 0.5\% premium as income (using the \% gain as a surrogate indicator for minimizing chances of assignment)! So, as an autodidact who had been successfully employing portfolio overwriting for FIRE and decided to write a book on it that was... vindicating.\\

To Nassim Nicholas Taleb. Your writing changed the way I think about the world. Not only via the introduction of new ideas, but it also galvanized a plethora of intuitions I already held. The sometimes gruff delivery and strong moral compass was not foreign or off-putting to me, as my father was a biker in the 70s. I'm currently working at a state institution and writing a book, so really taking your advice on barbelling. If I make it to FIRE, the first thing I'm doing is getting a degree in math. Hopefully I'll see you at the Real World Risk Institute someday.\footnote{Likely NNT will never see this, and with him being an applied expert in higher-level derivatives trading, I would probably be embarassed if he did. What I'm doing is basically drawing with crayons while he is painting the sistine chapel.}\\

To Ed Thorp, who had the audacity to see a card game people had been trying to beat for centuries and take an honest shot at it... and do it. One should always try. Even if you end up like Icarus, I'm sure you get a once-in-a-lifetime view before you get too close to the sun. The application of the Kelly Criterion (a formula that can guage the best size for a bet) to finance remains simultaneously one of your greatest and (somehow) least well-known contributions to the field of risk (perhaps due to technical difficulty?). My first endeavor after this book is to try to understand it more innately, either through coding, maths, or both.\\

\epigraph{\emph{"How will I know limits from lies, if I never try?”}}{---Thrice, \emph{The Melting Point of Wax},\\ from the album \emph{The Artist in the Ambulance}}

Ole Peters- I am looking forward to your book, and hope to see more episodes of Ergodicity TV soon!\\

To Grant Sanderson (aka 3blue1brown) who has a love for teaching, and the ability to make maths beautiful and unintimidating. You are a national treasure, and were kind enough to email me back when this book was in its early stages.\\

To Ross Enamait, who made the best books on fitness that ever existed as a labor of love and a service to people looking to become athletic anywhere, with any amount of equipment (including none). I hope that you would find this book to be as standalone and gimmick-free as your material.\\

To Jacob Lund Fisker, whose fantastic book \emph{Early Retirement Extreme} (also formatted in \LaTeX) served as inspiration for me.\\

To FIRE bloggers, especially Financial Velociraptor, who retired with less than 25x but maintained his income via options trading (including a simple and ingenious UVXY put strategy, that unfortunately no longer worked once the leverage of UVXY was reduced). He posted his positions and monthly finances in a way that was so simple and transparent that no other finance blogger has ever even come close.\footnote{His site is now sadly defunct, but he still posts transparencies on Facebook.}\\

To Lane, the investment blogger and author of Reminiscences of a Stockblogger. His website was another motivation for me to make something myself. A good (internet) friend and very insightful guy, applying high level, classical valuation to his area of expertise. I will never be as good as him, and I hope he writes a book someday. It's a passion project for him, but lots of investors out there would likely benefit from learning what he knows. His site is private to keep comments engaged and quality, but you can request access directly via \url{https://reminiscencesofastockblogger.com/}.

I'm also grateful to all the other thinkers that are iconoclastic and well-intentioned, especially and specifically Hamilton Morris. For a large part of my career I have dealt with terminally ill patients, face-to-face for the first 5 years of my career. This can have profound effects on a person. Realizing life is short and you would not like to spend the majority of it working is one of them. A true sense of the despair and hopelessness thanatophobia brings to the people unable to process and accept it is another.\\

To wit, I am hopeful for the use of the variety of pharmaceutical compounds generically referred to as psychedelics for treating that, and the myriad of other mental disorders our society is currently facing. Psilocybin for anxiety and depression in terminally ill patients, MDMA for PSTD, and ketamine for treatment resistant depression and alcohol abuse are the forefront of the field, with a stream of novel compounds in no short supply. I'm hopeful not only for patients, but the people I personally know who have benefited from these compounds, or could have before their lives were cut short.\footnote{I hope to live to see the day these compounds are available to “healthy normals.”}\\

Without discovering Hamilton's work, I would not have a well-educated or in depth viewpoint on the history, potential, medicinal chemistry and pharmacology of these compounds. More importantly, I wouldn't have a reasonable and logical view of the topic. Mr. Morris possesses the otherworldly skill of maintaining an objective and non-judgemental view of what is an emotionally, politically, and morally charged subject, which is truly the sign of a good journalist and anthropologist. The value of a de facto ethnography was something I underestimated when I was younger, foolishly favoring the “hard” sciences and being too dismissive of the power our culture plays in our life.\\ 

On top of that, he's a compelling writer and videographer, and all 3 seasons of his mini-documentary series Hamilton's pharmacopeia are beautifully produced. Each episode tells stories far above and beyond the standard "drug" show. I really hope someone reading this checks them out. Season 2 Episode 1 and Season 3 Episode 1 are a tour de force when viewed in succession. Also, he's extremely funny (a master of understated humor) and provides a very sonorous baritone contralto narration, as well as the real chemistry of how to synthesize pharmacologically active substances. \footnote{He has psychoactive toad (bufo alvaris) t shirts for sale as well, with all the proceeds going to Parkinson's research and Sonoran Desert Toad habitat preservation- they make great gifts!}

One example of a mind that was lost too soon to depression and alcoholism, who may still have been with us today were it not for the unavailability of a viable treatment as mentioned above was David Foster Wallace. I include him for two reasons- first, his Kenyon University commencement speech "This is Water" is something I listen to at least monthly (something I learned from Peter Attia) to remind me that as Dylan says, "you gotta serve somebody."\footnote{I prefer the Pop Staples version. Hoodoo Soul Band arguably does the best live performance on the planet.} Secondly, his way of thinking and ability to crystallize and convey it was singular.\\

I initially thought Infinite Jest was "just ok," but once I started to watch interviews of him to understand the book better, I had the realization that he had the insight and predictive capacity to perceive cultural issues ongoing and incoming way before they materialized. DFW's 2003 unedited interview for the German TV station ZDF made me realize for the first time that I could actually learn the most from hearing people who were way smarter than me in some dimension opine on things adjacent to (or altogether different) from their primary expertise.\footnote{Also, for better or for worse, he removed my shame surrounding gigantic footnotes.}\\

To the Universal Basic Income (UBI) proponents- Rutger Bregman, David Graeber (RIP), Michael Munger\footnote{If you would like to understand rhetoric and logic better, any podcast with Munger is a masterclass. I fear that sometimes people don't fully grasp the intent behind his style, though. For example, he often tells the tale of throwing a coke can in the trash during his lectures (much to the dismay of his environmentally-conscious students) to make a point about recycling, but admittedly digs it out after the lecture is over.}, and Andrew Yang. Politics aside (I'm not looking for a fight) I'm a big proponent of the idea of Universal Basic Income.\footnote{I readily admit that the biggest problem is we simply don't know what would happen if it were to be implemented nationally. It's an unknown unknown.} Like most people, I was initially opposed to it the first time I heard about it, but the more I learned, the more it made sense to me. Hopefully we have a large-scale trial at some point, but until then I'm hoping some people reading this book with the means to implement portfolio overwriting can have their own personal UBI fund, and maybe use some of the freedom it affords to make the world a better place.\footnote{I use some of my monthly premiums from selling calls to donate to Humanity Forward, a pro-UBI organization that splits funds between direct cash transfers and pro-UBI activity.}\\

Jonathan Laroquette and Seth Romatelli of Uhh Yeah Dude and The McElroy Brothers Griffin, Justin and Travis\footnote{The Adventure Zone's First Season, The Balance Arc, started out as a joke with a lot of low-brow humor and morphed into one of the most beautiful stories ever told. Worth your time.}-- thanks for bringing me lots of joy, and keeping me from being too serious. Laughter is carbonated holiness.\\ 

Kate Ma\footnote{Contact \url{signskate8@gmail.com} for commissions.} for colorization, cartoonization, and of illustrations and various thinkers.\\

Amy Pielow for on-demand help with \LaTeX\\

Finally, thanks to Curo and Shade for being good kitties.\\

Email me at \url{calltofirewizard@gmail.com}